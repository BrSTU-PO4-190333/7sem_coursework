\phantomsection
\addcontentsline{toc}{section}{ВВЕДЕНИЕ}
\section*{ВВЕДЕНИЕ}

Целью данного курсового проекта является обеспечение хранения данных при решении комплекса задач
для ИС «Косметический салон».
Задачей данного курсового проекта является изучения методики проектирования баз данных
для подсистемы «Инвентаризация».

\textbf{Проектирование баз данных} - процесс создания схемы базы данных
и определения необходимых ограничений целостности.
Для проектирования базы данных необходимо изначально построить модель системы,
которая будет проектироваться.

\textbf{Инвентаризация} - это проверка наличия имущества организации и состояния
её финансовых обязательств на определённую дату путём сличения фактических данных
с данными бухгалтерского учёта.
Это основной способ фактического контроля за сохранностью имущественных ценностей и средств.

\textbf{Модель} - это описание, абстрактное представление реальности в какой-либо форме
(математической, физической, символической, графической или дескриптивной),
предназначенное для представления определённых аспектов этой реальности
и позволяющее получить ответы на изучаемые вопросы.

\textbf{Модель деятельности организации} - совокупность взаимосвязанных и взаимодополняющих моделей различных типов,
представленных как правило в графической форме,
каждая из которых описывает существующую ситуацию в конкретной предметной области через ее структуры и функции.
Таким образом, чтобы создать автоматизированную систему и базу данных, как нижний уровень АС,
необходимо разработать модель, адекватно описывающую данный объект автоматизации.

\textbf{Объект автоматизации} - торговая деятельность салона, включающий комплекс задач по поставкам,
приему, хранению, реализации услуг.

Для разработки модели организации (организационной, функциональной, информационной и бизнес-процессов)
использовалась методология ARIS.

\textbf{ARIS} (Architecture of Integrated Information Systems) - методология моделирования бизнес-процессов организаций.
Любая организация в методологии ARIS рассматривается с пяти точек зрения:
организационной, функциональной, обрабатываемых данных, структуры бизнес-процессов,
продуктов и услуг, при этом каждая точка зрения представляет собой модель.

\textbf{В первом разделе} описываются разработанные для ОА <<Косметический салон>> модели,
а именно: организационная, функциональная, информационная и модель бизнес-процессов,
каждая из которых является важным шагом в проектировании базы данным,
способной обеспечить хранение данных, необходимых для решения задач ОА <<Косметический салон>>.
В первой главе построена организационная структура по нотации Organizational chart методологии ARIS,
из которой построена функциональная дерево по нотации Process landscape по методологии ARIS.
Разработаны макеты справочников, документов. Построена общая диаграмма связи по нотации General diagram методологии ARIS.
По функциональному дереву построен бизнес процесс по нотации Business process используя методологию ARIS.

\textbf{Во втором разделе} описывается разработка базы данных.
Данный процесс состоит из: создания концептуальной модели, основанной на моделях, описанных в первом разделе;
создания логической, а далее и физической модели на основании концептуальной.
Во второй главе разработаны локальные концептуальные модели по нотации Data model методологии ARIS,
которые были объедены в общую концептуальную модель.
По концептуальной модели построены механизмы целостности и логическая модель в SQL Power Arhitect,
который создал нам функциональную модель для СУБД Postgres.

\textbf{Эталоны} разработанные в первой главе занесены в приложение тестовых заданий для проверки.

\textbf{Результаты} занесенных эталонов в базу данных размещены в приложении результатов загрузки и проверки БД.
 
\newpage
