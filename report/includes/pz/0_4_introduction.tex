\phantomsection
\addcontentsline{toc}{section}{ВВЕДЕНИЕ}
\section*{ВВЕДЕНИЕ}

%Цели-задачи-метод-результаты

\textbf{Целью} курсового проекта является проектирование базы данных для подсистемы <<Инвентаризация>> информационной системы <<Косметический салон>>.

Инвентаризация - это проверка наличия имущества организации и состояния
её финансовых обязательств на определённую дату путём сличения фактических данных
с данными бухгалтерского учёта.
Это основной способ фактического контроля за сохранностью имущественных ценностей и средств.

\textbf{Задачами} данного курсового проекта является обследование объекта (ОА) автоматизации подсистемы <<Инвентаризация>> информационной системы <<Косметический салон>>.

При обследовании ОА использовалась методология ARIS построения процессной модели, которая строится на основе организационной функции,
информационной модели в единый бизнес-процесс

ARIS (Architecture of Integrated Information Systems) - методология моделирования бизнес-процессов организаций.
Любая организация в методологии ARIS рассматривается с пяти точек зрения:
организационной, функциональной, обрабатываемых данных, структуры бизнес-процессов,
продуктов и услуг, при этом каждая точка зрения представляет собой модель.

Для создания БД построена концептуальная, логическая и физическая модель. Физическая модель перенесена на СУБД Microsoft SQL Server 2022.

% В ходе курсового проекта используется \textbf{методы} ARIS Express:
% нотация для представления организационной структуры - organizational chart;
% нотация для представления функциональной модель - process landscape;
% нотация для представления информационной модели - general diagram;
% нотация для представления модели бизнес-процесса - business process;
% нотация для представления концептуальной модели - data model.

\textbf{Результатом} проектирования является реализованная база данных, которая хранит данные о приказе создания инвентаризационной комиссии,
данные вносимые при проведении инвентаризации, которая может генерировать акт об инвентаризации.

% Целью данного курсового проекта является обеспечение хранения данных при решении комплекса задач
% для ИС «Косметический салон».
% Задачей данного курсового проекта является изучения методики проектирования баз данных
% для подсистемы «Инвентаризация».

% \textbf{Проектирование баз данных} - процесс создания схемы базы данных
% и определения необходимых ограничений целостности.
% Для проектирования базы данных необходимо изначально построить модель системы,
% которая будет проектироваться.

% \textbf{Инвентаризация} - это проверка наличия имущества организации и состояния
% её финансовых обязательств на определённую дату путём сличения фактических данных
% с данными бухгалтерского учёта.
% Это основной способ фактического контроля за сохранностью имущественных ценностей и средств.

% \textbf{Модель} - это описание, абстрактное представление реальности в какой-либо форме
% (математической, физической, символической, графической или дескриптивной),
% предназначенное для представления определённых аспектов этой реальности
% и позволяющее получить ответы на изучаемые вопросы.

% \textbf{Модель деятельности организации} - совокупность взаимосвязанных и взаимодополняющих моделей различных типов,
% представленных как правило в графической форме,
% каждая из которых описывает существующую ситуацию в конкретной предметной области через ее структуры и функции.
% Таким образом, чтобы создать автоматизированную систему и базу данных, как нижний уровень АС,
% необходимо разработать модель, адекватно описывающую данный объект автоматизации.

% \textbf{Объект автоматизации} - торговая деятельность салона, включающий комплекс задач по поставкам,
% приему, хранению, реализации услуг.

% Для разработки модели организации (организационной, функциональной, информационной и бизнес-процессов)
% использовалась методология ARIS.

% \textbf{ARIS} (Architecture of Integrated Information Systems) - методология моделирования бизнес-процессов организаций.
% Любая организация в методологии ARIS рассматривается с пяти точек зрения:
% организационной, функциональной, обрабатываемых данных, структуры бизнес-процессов,
% продуктов и услуг, при этом каждая точка зрения представляет собой модель.

\textbf{В первом разделе} описываются разработанные для объекта автоматизации подсистемы <<Инвентаризация>> информационной системы <<Косметический салон>>
организационная модель, функциональная модель, информационная модель и модель бизнес-процессов.
Организационная модель построена с помощью методологии ARIS нотации <<organizational diagram>>.
Функциональная модель, которая соответствует организационной модели, построена с помощью методологии ARIS нотации <<process landscape>>.
Информационная модель построена с помощью методологии ARIS нотации <<general diagram>>, которая показывает связи
между разработанными макетами справочных и оперативных документов.
Модель бизнес-процесса, которая соответствует функциональной модели, построена с помощью методологии ARIS, нотации <<business process>>. 

% В первой главе построена организационная структура по нотации Organizational chart методологии ARIS,
% из которой построена функциональная дерево по нотации Process landscape по методологии ARIS.
% Разработаны макеты справочников, документов. Построена общая диаграмма связи по нотации General diagram методологии ARIS.
% По функциональному дереву построен бизнес процесс по нотации Business process используя методологию ARIS.

\textbf{Во втором разделе} описывается концептуальная модель, логической модель и физическая модель.
Концептуальная модель основана на информационной модели и построена с помощью методологии ARIS нотации data model.
Концептуальная модель строится для каждого документа и объединяется в общую концептуальную модель.
Логическая модель основана на общей концептуальной модели и строится с помощью SQL Power Architect.
Физическая модель - результат логической модели, а именно SQL-скрипты для системы управления базы данных Microsoft SQL Server 2022,
которые сгенерирован SQL Power Architect и редактирован разработчиком.

% Данный процесс состоит из: создания концептуальной модели, основанной на моделях, описанных в первом разделе;
% создания логической, а далее и физической модели на основании концептуальной.
% Во второй главе разработаны локальные концептуальные модели по нотации Data model методологии ARIS,
% которые были объедены в общую концептуальную модель.
% По концептуальной модели построены механизмы целостности и логическая модель в SQL Power Arhitect,
% который создал нам функциональную модель для СУБД Postgres.

\textbf{В приложении А} занесены тестовые задания для проверки базы данных: заполненные макеты справочных и оперативных документов в виде таблиц.

% \textbf{Эталоны} разработанные в первой главе занесены в приложение тестовых заданий для проверки.

\textbf{В приложении Б} занесены SQL INSERT-скрипты и транзакции c INSERT-командами для загрузки данных в базу данных
и результаты загрузки базы данных представленные в виде скриншотов выборок SELECT-команд.

% \textbf{Результаты} занесенных эталонов в базу данных размещены в приложении результатов загрузки и проверки БД.
 
\newpage


% Целью данного курсового проекта является спроектировать базу данных для обеспечения хранения данных при решении задачи инвентаризации для информационной системы «Косметический салон». 

% Проектирование баз данных — процесс создания схемы базы данных и определения необходимых ограничений целостности. Для проектирования базы данных нужно сначала построить модель проектируемой системы.

% Модель базы данных – это тип модели данных, которая определяет логическую структуру базы данных и в корне определяет, каким образом данные могут храниться, организовываться и обрабатываться. Таким образом, чтобы создать автоматизированную систему и базу данных, необходимо разработать модель, корректно описывающую данный объект автоматизации.

% Объект автоматизации (ОА) – комплекс задач по реализации услуг косметического салона.

% Задачей данного курсового проекта является обследование объекта автоматизации (ОА), проектирование концептуальной модели через модель «сущность-связь», проектирование логической модели на основе концептуальной модели через совокупность реляционных отношений и проектирование физической модели на основе логической модели, используя совокупность объектов, поддерживаемых выбранной СУБД. 

% Обследование ОА включает в себя сбор, систематизацию и анализ исходных данных об ОА, использование полученной информация для создания организационной, функциональной, информационной и бизнес-процессной моделей.

% Для разработки таких частных моделей как организационная, функциональная, информационная, процессная модели, а также для разработки концептуальной модели, была использована методология ARIS.

% ARIS (Architecture of Integrated Information Systems) — методология для моделирования бизнес-процессов организаций. Любая организация в методологии ARIS рассматривается с пяти точек зрения: организационной, функциональной, обрабатываемых данных, структуры бизнес-процессов, продуктов и услуг. Каждая точка зрения представляет собой модель. 

% При разработке частных моделей были использованы следующие нотации: Process Landscape методологии ARIS для проектирования функциональной модели, General Diagram методологии ARIS для проектирования информационной модели, Extended Events Process Chain методологии ARIS для проектирования бизнес-процессной модели и Organizational Chart методологии ARIS для проектирования организационной модели.

% Для разработки логической и физической моделей были использованы CASE-средства, предоставляемые пакетом ERWin Data Modeler.

% ERwin Data Modeler  — это компьютерная программа для проектирования и документирования баз данных.

% Результатом выполнения курсового проекта является спроектированная база данных для выбранного объекта автоматизации с возможностью формирования, ведения и документирования справочников, а также отчетных документов.
