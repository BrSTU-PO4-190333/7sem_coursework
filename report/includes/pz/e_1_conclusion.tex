\phantomsection
\addcontentsline{toc}{section}{ЗАКЛЮЧЕНИЕ}
\section*{ЗАКЛЮЧЕНИЕ}

В ходе выполнения данного курсового проекта была спроектирована база данных
для подсистемы <<Инвентаризация>> информационной системы <<Косметический салон>>.
Цели и задачи были реализованы полностью.

При выполнении применялась методология моделирования бизнес-процессов ARIS.
Использовался инструмент моделирования ARIS Express.
Был спроектирован бизнес-процесс.
От бизнес процессов были построены локальные концептуальные модели.
Локальные концептуальные модель были объедены в одну концептуальную модель.
Программа для проектирования физической модели - ERwin.
После создания общей концептуальной модель создавали физическую модель с указанием
типа, домена, ключа первичного или внешнего.
Использовали СУБД MySQL для создания, проверки и загрузки базы данных.
Из программы SQL Power Architect экспортировали скрипты для создания таблиц в базе данных.
После успешного создания таблиц, на основании эталонов, создали INSERT SQL запросы.

\newpage
