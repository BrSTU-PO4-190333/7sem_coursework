\phantomsection
\addcontentsline{toc}{section}{ЗАКЛЮЧЕНИЕ}
\section*{ЗАКЛЮЧЕНИЕ}

В ходе выполнения данного курсового проекта была спроектирована база данных
для подсистемы <<Инвентаризация>> информационной системы <<Косметический салон>>.
Цели и задачи были реализованы полностью.

При выполнении применялась методология моделирования бизнес-процессов ARIS.
Использовался инструмент моделирования ARIS Express 2.4i.

В ходе работы была составлена организационная модель по нотации organizational chart методологии ARIS.
Из организационной модели построена функциональная модель по нотации Proccess landscape методологии ARIS.
Были разработаны эталоны и макеты справочников и документов.
Разработаны схемы связей документов по по нотации General diagram методологии ARIS.
Из функциональной модели построен бизнес процесс по нотации Business process методологии ARIS.

В ходе курсовой работы разработаны локальные концептуальные модели по нотации Data model методологии ARIS,
которые объедены в общую концептуальную модель. 
Из общей концептуальной модели создавали механизмы целостности и спроектировали
логическую модель в SQL Power Architect 1.0.7, которая доведена до третьей нормальной формы.
Из логической модели сделана физическая модель с получением скриптов для создания
таблиц в СУБД Postgres в SQL Power Architect 1.0.7.

После успешного создания таблиц, на основании эталонов, создали INSERT SQL запросы,
подтвердив правильность проектирования базы данных.

\newpage
