\phantomsection
\addcontentsline{toc}{section}{ЗАКЛЮЧЕНИЕ}
\section*{ЗАКЛЮЧЕНИЕ}

В ходе выполнения данного курсового проекта была спроектирована база данных
для подсистемы <<Инвентаризация>> информационной системы <<Косметический салон>>.
Цели и задачи были реализованы полностью.

При выполнении применялась методология моделирования ARIS.
Использовался инструмент моделирования ARIS Express 2.4i.

В ходе выполнения курсовой работы была составлена организационная модель по методологии ARIS нотации <<organizational chart>>.
Из организационной модели построена функциональная модель по методологии ARIS нотации <<proccess landscape>>.
Были разработаны эталоны и макеты справочников и оперативных документов.
Разработаны схемы связей документов по методологии ARIS нотации <<general diagram>>.
Из функциональной модели построен бизнес процесс по методологии ARIS нотации <<business process>>.

В ходе выполнения курсовой работы были разработаны локальные концептуальные модели по методологии ARIS нотации <<data model>>,
которые объедены в общую концептуальную модель. 
Из общей концептуальной модели проектировали
логическую модель в SQL Power Architect 1.0.7, которая была доведена до третьей нормальной формы.
Из логической модели сделана физическая модель с получением скриптов для создания
таблиц в СУБД Microsoft SQL Server в SQL Power Architect 1.0.7.

После успешного создания таблиц, на основании эталонов, создали INSERT SQL-запросы для
заполнения справочников и транзакции с INSERT-командами для заполнения оперативных документов с табличной частью.
Скрипты отработали без ошибок, что говорит, что база данных спроектирована верно.

\newpage
